\documentclass[titlepage]{article}
\usepackage[12pt]{extsizes} 
\usepackage[T2A]{fontenc}
\usepackage[utf8]{inputenc}
\usepackage[english,russian]{babel}
%\usepackage{pscyr}
\usepackage{hyperref}
\usepackage{setspace}
\usepackage{amsmath,amssymb,amsfonts,amsthm,secdot}
\usepackage[left=30mm, top=20mm, right=30mm, bottom=20mm, nohead, footskip=15mm]{geometry} 
\usepackage[pdftex]{graphicx}
\usepackage[indentfirst]{titlesec}
\usepackage[usenames]{color}
\usepackage{colortbl}
\usepackage{listings}
\usepackage{secdot}

\def\l{\left}
\def\r{\right}
\def\le{\leqslant}
\def\ge{\geqslant}

\begin{document} 

\newtheorem{theorem}{Теорема}
\newtheorem{lemma}{Лемма}
\newtheorem{definition}{Определение}
\renewcommand{\proofname}{Доказательство}

\begin{center}
\hfill \break
\hfill \break
\hfill \break
\LARGE Вычисление несобственного интеграла с помощью квадратурных формул \\
\hfill \break
\large Д.А. Михайлин \\
\hfill \break
\today \\

\end{center}

\section{Постановка задачи}
Требуется приближенно вычислить интеграл: 
\begin{gather}
	\int_{0}^{+\infty}(1 + \frac{1}{x})\frac{(sin{3x})^2}{\ln(1 + x)}e^{-x^3}dx
\end{gather}
с заданной точностью $\varepsilon = 10^{-2}$.

\section{Доказательство сходимости интеграла}
Прежде чем приступить к вычислению интеграла, необходимо убедиться в его сходимости во всей области интегрирования. Для этого исследуем подынтегральную функцию $f(x) = (1 + \frac{1}{x})\frac{(sin{3x})^2}{ln(1 + x)}e^{-x^3}$ в особых точках на промежутке $[0, +\infty)$. Рассмотрим две особые точки: $x = 0$,  $x \to \infty$:

\begin{enumerate}
	\item $x = 0$ \\
	\begin{gather}
		\notag \lim_{x \to 0}{f(x)} = \lim_{x \to 0}{(1 + \frac{1}{x})\frac{(sin{3x})^2}{\ln(1 + x)}e^{-x^3}} = \\
		\notag = \lim_{x \to 0}{\frac{9x^2 + o(x^2)}{x(x + o(x))}(1 + x)(e^{-x^2})} = 9
	\end{gather}
	Таким образом, $x = 0$ является устранимой особой точкой. Это значит, что в нуле мы можем доопределить $f(x)$ до непрерывной функции, поэтому интеграл не является несобственным в точке $x = 0$.
\item $x \to \infty$ \\
	
	$\exists$ такое большое число $C$, что:
	\begin{gather}
		\notag \l| \int_{C}^{+\infty}{f(x)dx} \r| \le \int_{C}^{+\infty}{\l| \frac{x + 1}{x^2}\frac{sin^2{3x}}{\ln{(1+x)}}e^{-x^2}\r|dx^2} \le \\ \notag \le\int_{C}^{+\infty}{ 2e^{-x^3} dx^2} = 2e^{-C^2} < \infty
	\end{gather}
Сходимость интеграла на бесконечности доказана.
\end{enumerate}
\section{Оценки параметров $\delta_1$, $C$}
Положим $\varepsilon_1 = \varepsilon_2 = 2\cdot10^{-3}$ $\varepsilon_3 = 6\cdot10^{-3}$ и разобьем наш промежуток интегрирования на 3 части: $[0, \delta_1), [\delta_1, C), [C, +\infty)$.

Теперь нам необходимо подобрать числа $\delta_1$ и $C$ таким образом, чтобы $\l|\int_{0}^{\delta_1}{f(x)dx}\r| \le \varepsilon_1$ и $\l|\int_{C}^{+\infty}{f(x)dx}\r| \le \varepsilon_3$. Тем самым, задача сведется к вычислению интеграла $\tilde I_2 = \int_{\delta_1}^{С}{f(x)dx}$ с точностью $\varepsilon_2$.	
\begin{enumerate}
	\item Оценим $\delta_1$. \\
	Будем пользоваться следующими фактами:

	\begin{gather}
		\notag \sin(x) \le x  \\ 
		\notag \text{и} \\
		\notag \frac{x}{x + 1} \le \ln{(x + 1)}
	\end{gather}

	\begin{gather}
		\notag \l|\int_{0}^{\delta_1}{f(x)dx}\r| \le \int_{0}^{\delta_1}{\l|f(x)\r|dx} \le \int_{0}^{\delta_1}{\l| (1 + \frac{1}{x})\frac{(sin{3x})^2}{\ln(1 + x)}e^{-x^3}\r| dx} \le \\
		\notag \le \int_{0}^{\delta_1}{\l| \frac{x + 1}{x}e^{-x^2}\frac{x}{x + 1}9x^2\r|dx} =
		\notag \le \int_{0}^{\delta_1}{9x^{2}dx} = 3\delta_1^{3} \le \varepsilon_1
	\end{gather}
	Следовательно, $$\delta_1 \le \l(\frac{1}{3}\varepsilon_1\r)^{1/3}$$
	Подставив $\varepsilon_1 = 2\cdot10^{-3}$, получим $\bf{\delta_1 = 0.087}$.
	\item Оценим $C$.
		\begin{gather}
		\notag \l|\int_{C}^{+\infty}{f(x)dx}\r| \le \int_{C}^{+\infty}{\l| (1 + \frac{1}{x})\frac{(sin{3x})^2}{\ln(1 + x)}e^{-x^3}\r|dx} \le \\
		\notag \le \int_{C}^{+\infty}{\l|(\frac{x + 1}{x^2})\frac{(sin{3x})^2}{\ln(1 + x)}e^{-x^3}\r|dx^2} \le \frac{C + 1}{2C^2}\frac{1}{\ln{(1 + C)}}\int_{C}^{+\infty}{ e^{-x^2} dx^2} = \frac{C + 1}{2C^2}\frac{1}{\ln{(1 + C)}}e^{-C^2} = \\F(C) \le \varepsilon_3 \\
	F(2) > 0.0062
	\end{gather}
	Покажем, что можем взять C = 2.
	\begin{gather}
		\notag \l|\int_{2}^{+\infty}{f(x)dx}\r| \le \int_{2}^{3}{ (1 + \frac{1}{x})\frac{(sin{3x})^2}{\ln(1 + x)}e^{-x^3}dx} + \int_{3}^{+\infty}{ (1 + \frac{1}{x})\frac{(sin{3x})^2}{\ln(1 + x)}e^{-x^3}dx}  \\ \le \int_{2}^{3}{ (1 + \frac{1}{x})\frac{(sin{3x})^2}{\ln(1 + x)}e^{-x^3}dx} + 0.00005  \le 0.0055 + 0.0005 < \varepsilon_3
	\end{gather}
\end{enumerate}
\end{document}